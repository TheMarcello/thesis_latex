%%% APPENDIX
\chapter{Appendix}\label{chap:appendix}

Additional explanations:
\marginpar{\flushleft make into actual Appendix type}
\section{Complete list of DUTs}

\begin{landscape}

\begin{table}[h]
    \caption{Complete list of the tested devices with additional details}
    \label{tab:full_devices_tested}
    \footnotesize
        \begin{tabularx}{\textheight}{|l|l|l|l|l|l|l|l|X|}
            \hline
            \textbf{Device name} & \textbf{Vendor} & \textbf{Sensor ID} &  \begin{tabular}{@{}l@{}}\textbf{Pads,} \\ \textbf{used channels}\end{tabular} & \begin{tabular}{@{}l@{}}\textbf{Fluence} \\ $[n_{eq}/\si{cm^2}]$ \end{tabular} &\begin{tabular}{@{}l@{}} \textbf{Radiation} \\ \textbf{type} \end{tabular} & \textbf{Board name} & \begin{tabular}{@{}l@{}}\textbf{Board} \\ \textbf{channels} \end{tabular} & \textbf{Notes} \\
            \hline
            CNM-W4 & CNM & CNM-R15973-W4-D168 & single & 0 & - & JSI-B12 & - & reference \\ 
            CNM-W5 & CNM & CNM-R15973-W5-D138 & single & 0 & - & JSI-B14 & - & reference \\ 
            CNM-W5-1.5E15 & CNM & CNM-R15973-W5-D29 & single & $\num{1.50E+15}$ & neutron & JSI-B5 & - & \\ 
            CNM-W3-2.5E15 & CNM & CNM-R15973-W3-D29 & single & $\num{2.50E+15}$ & neutron & JSI-PP1 & - & \\
            USTC2.1-W17 & USTC & USTC2.1-W17-P6-A-2x2 & 2x2, 2 channels & 0 & - & CERN-3 & Ch1,Ch2 & \\
            USTC2.1-W19 & USTC & USTC2.1-W19-P5-A-1x1 & single & 0 & - & CERN-3 & - & not tested \\
            USTC2.1-W17-2E14 & USTC & USTC2.1-W17-P6-A-2x2 & 2x2, 1 channel & 0 & - & JSI-B2 & - & missing \\ 
            IMEv3-W12-2x2 & IHEP & IMEv3-W12-C2-2-2 & 2x2, 2 channels & 0 & - & CERN-1 & Ch1,Ch2 &  \\ 
            IMEv3-W12-1x3 & IHEP & IMEv3-W12-C3-1-4(and5) & 1x3, 2 channels & 0 & - & CERN-1 & Ch3,Ch4 &  \\ 
            IMEv3-W12-2x2-1.5E15 & IHEP & IMEv3-W12-B2-2-9-1 & 2x2, 3 channels & $\num{1.50E+15}$ & neutron & CERN-2 & Ch0,Ch1,Ch2 &  \\ 
            IMEv3-W16-1x3-1.5E15 & IHEP & IMEv3-W16-Q4-D4-1-4 & 1x3, 1 channel & $\num{1.50E+15}$ & neutron & CERN-2 & Ch3 &  \\ 
            IMEv2-W7-1E14 & IHEP & W7-II-C2-1-7IMEv2-W7Q2 & single & $\num{1.00E+14}$ & proton & JSI-B6 & - &  \\ 
            IMEv2-W7-6.5E14 & IHEP & W7-II-C2-1-7IMEv2-W7Q2 & single & $\num{6.50E+14}$ & proton & JSI-PP4 & - &  \\ 
            IMEv3-W16-8E14 & IHEP & IHEP-IMEv3-W16-Q4-D3-1-4 & single & $\num{8.00E+14}$ & proton & JSI-B7 & - &  \\ 
            IMEv3-W16-2.5E15 & IHEP & IHEP-IMEv3-W16-Q4-E3-1-4 & single & $\num{2.50E+15}$ & neutron & JSI-B13 & - &  \\ 
            \hline
        \end{tabularx}
\end{table}
\end{landscape}


\section{Empty vertical line}
Due to one dead column of pixels of a MIMOSA \marginpar{\flushleft add plots of the DATA showing the vertical line, or link/ref to it} some batches presented an empty vertical line. The cause of this was found in one of the MIMOSA planes, which had a dead column of pixels and interfered with the track reconstruction. This problem turned out to be inconsequential for the analysis, except for the very low statistics in that very small region.
\begin{figure}[!ht]
    \centering
    \includegraphics[width=.9\linewidth]{Images/appendix/hits_MIMOSA4.png}
    \caption{The hits heatmap of the MIMOSA plane n°4, which shows clearly the source of the "empty line" that appeared in the data.}
    \label{fig:MIMOSA4_hits}
\end{figure}


\section[Vavilov vs Landau distribution]{On the Vavilov distribution approximation}\label{sec:vavilov_vs_landau_distribution}

% \cite[189]{NAP20066}
%     
% \maketitle
    
\begin{tcolorbox}[breakable, size=fbox, boxrule=1pt, pad at break*=1mm,colback=cellbackground, colframe=cellborder]
\prompt{In}{incolor}{1}{\boxspacing}
\begin{Verbatim}[commandchars=\\\{\}]
\PY{k+kn}{import} \PY{n+nn}{numpy} \PY{k}{as} \PY{n+nn}{np} \PY{c+c1}{\PYZsh{} NumPy}
\PY{k+kn}{from} \PY{n+nn}{scipy}\PY{n+nn}{.}\PY{n+nn}{optimize} \PY{k+kn}{import} \PY{n}{fsolve} 
\PY{k+kn}{from} \PY{n+nn}{IPython}\PY{n+nn}{.}\PY{n+nn}{display} \PY{k+kn}{import} \PY{n}{Image}
\PY{k+kn}{import} \PY{n+nn}{pandas} \PY{k}{as} \PY{n+nn}{pd} \PY{c+c1}{\PYZsh{} Pandas}
\PY{n}{pd}\PY{o}{.}\PY{n}{options}\PY{o}{.}\PY{n}{display}\PY{o}{.}\PY{n}{float\PYZus{}format} \PY{o}{=} \PY{l+s+s1}{\PYZsq{}}\PY{l+s+si}{\PYZob{}:5,.4E\PYZcb{}}\PY{l+s+s1}{\PYZsq{}}\PY{o}{.}\PY{n}{format}
\end{Verbatim}
\end{tcolorbox}

    \hypertarget{calculating-the-factor-the-determines-if-the-vavilov-distribution-can-be-approximated-by-a-landau}{%
\section{Calculating the factor the determines if the Vavilov
distribution can be approximated by a
Landau}\label{calculating-the-factor-the-determines-if-the-vavilov-distribution-can-be-approximated-by-a-landau}}

Bethe-Bloch mean energy loss: \[
\left\langle-\frac{dE}{dx} \right\rangle \frac{1}{\rho} =  K z_0^2 \frac{1}{\beta^2} \frac{Z}{A M_u} \cdot\left[\frac{1}{2}\ln \left(\frac{2m_e c^2 \beta^2 W_{max}}{I^2 \cdot (1-\beta^2)}\right) - \beta^2\right]
\]

\[
K/M_u = 4 \pi N_A r_e^2 m_e c^2 /M_u = 0.307075\space \text{MeV g}^{-1} \text{cm}^2
\]

\[
r_e  = \frac{e^2}{4\pi\varepsilon_0 m_e c^2} \\
M_u = 1 \frac{g}{mol} \quad z_0=1 \\
\left( W_{max} \equiv \epsilon_{max} \right)
\]

For \textbf{silicon}: \[
Z = 14 \\
A = 28.085 \\ 
\rho = 2.329085 \text{  g cm}^{-3}\\
I = 173 \text{ eV} % \quad \text{(mean excitation energy)}
\]

\hypertarget{calculating-beta-for-a-proton-of-energy-e_k}{%
\subsubsection{\texorpdfstring{Calculating \(\beta\) for a proton of
energy
\(E_k\)}{Calculating \textbackslash beta for a proton of energy E\_k}}\label{calculating-beta-for-a-proton-of-energy-e_k}}

\[
E_k = (\gamma - 1) m_o c^2 \\
\gamma = \frac{E_k + m_o c^2}{m_o c^2} \\
\beta^2 = 1 - \left(\frac{m_o c^2}{E_k +m_o c^2} \right)^2
\]

for a proton of energy \(E_k=120\space \text{GeV}\),
\(m_0 = 938.272\space \text{MeV/c}^2\)

    \begin{tcolorbox}[breakable, size=fbox, boxrule=1pt, pad at break*=1mm,colback=cellbackground, colframe=cellborder]
\prompt{In}{incolor}{2}{\boxspacing}
\begin{Verbatim}[commandchars=\\\{\}]
\PY{n}{energy} \PY{o}{=} \PY{l+m+mf}{120e3} \PY{c+c1}{\PYZsh{} MeV  (120 GeV)}
\PY{c+c1}{\PYZsh{} m\PYZus{}particle = 938.272088 \PYZsh{} MeV/c\PYZca{}2  protons }
\PY{n}{m\PYZus{}particle} \PY{o}{=} \PY{l+m+mf}{938.213} \PY{c+c1}{\PYZsh{} MeV/c\PYZca{}2     protons (older value)}

\PY{n}{gamma} \PY{o}{=} \PY{p}{(}\PY{n}{energy} \PY{o}{+} \PY{n}{m\PYZus{}particle}\PY{p}{)}\PY{o}{/}\PY{n}{m\PYZus{}particle}
\PY{n}{beta\PYZus{}2} \PY{o}{=} \PY{l+m+mi}{1} \PY{o}{\PYZhy{}} \PY{n}{m\PYZus{}particle}\PY{o}{*}\PY{o}{*}\PY{l+m+mi}{2}\PY{o}{/}\PY{p}{(}\PY{n}{m\PYZus{}particle} \PY{o}{+} \PY{n}{energy}\PY{p}{)}\PY{o}{*}\PY{o}{*}\PY{l+m+mi}{2}

\PY{n+nb}{print}\PY{p}{(}\PY{l+s+s2}{\PYZdq{}}\PY{l+s+s2}{ gamma: }\PY{l+s+s2}{\PYZdq{}}\PY{p}{,} \PY{n}{gamma}\PY{p}{,} \PY{l+s+s2}{\PYZdq{}}\PY{l+s+se}{\PYZbs{}n}\PY{l+s+s2}{\PYZdq{}}\PY{p}{,} \PY{l+s+s2}{\PYZdq{}}\PY{l+s+s2}{beta²: }\PY{l+s+s2}{\PYZdq{}}\PY{p}{,} \PY{n}{beta\PYZus{}2}\PY{p}{)}
\end{Verbatim}
\end{tcolorbox}

    \begin{Verbatim}[commandchars=\\\{\}]
 gamma:  128.90272571367058
 beta²:  0.999939816727599
    \end{Verbatim}

    from https://academic.oup.com/book/43645/chapter/365027598\#404941541 :

The parameter \(\kappa\) of the Vavilov distribution determines if it
becomes close to a Landau (\(\kappa\rightarrow0\)) or a Gaussian
(\(\kappa 10\)).

To calculate \(\kappa\): 
\[
\kappa = \frac{\xi}{\epsilon_{max}}
\]

where: \[
\xi = \frac{1}{2} K \frac{Z}{A} \frac{s} {\beta^2}
\]

\(s = \Delta x \cdot \rho\)

and from https://nap.nationalacademies.org/read/20066/chapter/10\#188 :
\[
\epsilon_{max} = \frac{2m_e c^2 \beta^2}{1-\beta^2}\left[ 1 + \frac{2m_e}{M}\frac{1}{\sqrt{1-\beta^2}} + \left(\frac{m_e}{M}\right)^2 \right]^{-1}
\]

where \(M\) is the mass of the incoming particle.

Also written as:

\[
\epsilon_{max} = \frac{2m_e c^2 \beta^2 \gamma^2} { 1 + \frac{2\gamma m_e}{M} + \left(\frac{m_e}{M}\right)^2}
\]

    \hypertarget{applying-this-to-a-thin-silicon-layer-50mu-textm}{%
\subsubsection{\texorpdfstring{Applying this to a thin silicon layer
(50\(\mu \text{m}\))}{Applying this to a thin silicon layer (50\textbackslash mu \textbackslash text\{m\})}}\label{applying-this-to-a-thin-silicon-layer-50mu-textm}}

(The masses are either multiplied by \(c^2\) or in a ratio, the mass
units used are \(\text{MeV}/c^2\))

    \begin{tcolorbox}[breakable, size=fbox, boxrule=1pt, pad at break*=1mm,colback=cellbackground, colframe=cellborder]
\prompt{In}{incolor}{3}{\boxspacing}
\begin{Verbatim}[commandchars=\\\{\}]
\PY{c+c1}{\PYZsh{}\PYZsh{}\PYZsh{} variables values}
\PY{n}{m\PYZus{}electron} \PY{o}{=} \PY{l+m+mf}{0.51099895} \PY{c+c1}{\PYZsh{} MeV/c\PYZca{}2}
\PY{n}{z\PYZus{}0} \PY{o}{=} \PY{l+m+mi}{1} \PY{c+c1}{\PYZsh{} electric charge of incoming particle}
\PY{n}{delta\PYZus{}x} \PY{o}{=} \PY{l+m+mf}{5e\PYZhy{}3} \PY{c+c1}{\PYZsh{} cm  (50um)}

\PY{c+c1}{\PYZsh{}\PYZsh{}\PYZsh{} silicon}
\PY{n}{Z} \PY{o}{=} \PY{l+m+mi}{14}   \PY{c+c1}{\PYZsh{} atomic number}
\PY{n}{A} \PY{o}{=} \PY{l+m+mf}{28.085}  \PY{c+c1}{\PYZsh{} atomic weight}
\PY{n}{density} \PY{o}{=} \PY{l+m+mf}{2.329085} \PY{c+c1}{\PYZsh{} g cm\PYZca{}\PYZhy{}3}
\PY{n}{mean\PYZus{}excitation} \PY{o}{=} \PY{l+m+mf}{173e\PYZhy{}6} \PY{c+c1}{\PYZsh{} MeV  (mean excitation energy of Silicon: 173 eV)}

\PY{n}{s} \PY{o}{=} \PY{n}{delta\PYZus{}x} \PY{o}{*} \PY{n}{density} \PY{c+c1}{\PYZsh{} g cm\PYZca{}\PYZhy{}2}
\PY{n}{K} \PY{o}{=} \PY{l+m+mf}{0.307075} \PY{c+c1}{\PYZsh{} MeV g\PYZca{}\PYZhy{}1 cm\PYZca{}2}
\end{Verbatim}
\end{tcolorbox}

    \begin{tcolorbox}[breakable, size=fbox, boxrule=1pt, pad at break*=1mm,colback=cellbackground, colframe=cellborder]
\prompt{In}{incolor}{4}{\boxspacing}
\begin{Verbatim}[commandchars=\\\{\}]
\PY{k}{def} \PY{n+nf}{xi\PYZus{}func}\PY{p}{(}\PY{n}{energy}\PY{p}{)}\PY{p}{:}
    \PY{n}{beta\PYZus{}2} \PY{o}{=} \PY{l+m+mi}{1} \PY{o}{\PYZhy{}} \PY{n}{m\PYZus{}particle}\PY{o}{*}\PY{o}{*}\PY{l+m+mi}{2}\PY{o}{/}\PY{p}{(}\PY{n}{m\PYZus{}particle} \PY{o}{+} \PY{n}{energy}\PY{p}{)}\PY{o}{*}\PY{o}{*}\PY{l+m+mi}{2}
    \PY{k}{return} \PY{p}{(}\PY{l+m+mi}{1}\PY{o}{/}\PY{l+m+mi}{2} \PY{o}{*} \PY{n}{K}\PY{p}{)} \PY{o}{*} \PY{p}{(}\PY{n}{Z} \PY{o}{/} \PY{n}{A}\PY{p}{)} \PY{o}{*} \PY{p}{(}\PY{n}{s} \PY{o}{/} \PY{n}{beta\PYZus{}2}\PY{p}{)}

\PY{k}{def} \PY{n+nf}{eps\PYZus{}max\PYZus{}func}\PY{p}{(}\PY{n}{energy}\PY{p}{)}\PY{p}{:}
    \PY{n}{gamma} \PY{o}{=} \PY{p}{(}\PY{n}{energy} \PY{o}{+} \PY{n}{m\PYZus{}particle}\PY{p}{)}\PY{o}{/}\PY{n}{m\PYZus{}particle}
    \PY{n}{beta\PYZus{}2} \PY{o}{=} \PY{l+m+mi}{1} \PY{o}{\PYZhy{}} \PY{n}{m\PYZus{}particle}\PY{o}{*}\PY{o}{*}\PY{l+m+mi}{2}\PY{o}{/}\PY{p}{(}\PY{n}{m\PYZus{}particle} \PY{o}{+} \PY{n}{energy}\PY{p}{)}\PY{o}{*}\PY{o}{*}\PY{l+m+mi}{2}
    \PY{k}{return} \PY{p}{(}\PY{l+m+mi}{2} \PY{o}{*} \PY{n}{m\PYZus{}electron} \PY{o}{*} \PY{n}{beta\PYZus{}2} \PY{o}{*} \PY{n}{gamma}\PY{o}{*}\PY{o}{*}\PY{l+m+mi}{2}\PY{p}{)} \PY{o}{/} \PY{p}{(}\PY{l+m+mi}{1} \PY{o}{+} \PY{l+m+mi}{2}\PY{o}{*}\PY{n}{gamma}\PY{o}{*}\PY{n}{m\PYZus{}electron}\PY{o}{/}\PY{n}{m\PYZus{}particle} \PY{o}{+} \PY{p}{(}\PY{n}{m\PYZus{}electron}\PY{o}{/}\PY{n}{m\PYZus{}particle}\PY{p}{)}\PY{o}{*}\PY{o}{*}\PY{l+m+mi}{2}\PY{p}{)} \PY{c+c1}{\PYZsh{} MeV}

\PY{k}{def} \PY{n+nf}{kappa\PYZus{}func}\PY{p}{(}\PY{n}{energy}\PY{p}{,} \PY{n}{kappa\PYZus{}value}\PY{o}{=}\PY{l+m+mf}{0.01}\PY{p}{)}\PY{p}{:}
    \PY{l+s+sd}{\PYZdq{}\PYZdq{}\PYZdq{}kappa as a function of the energy (to find the zeros for k=kappa\PYZus{}value)\PYZdq{}\PYZdq{}\PYZdq{}}
    \PY{n}{gamma} \PY{o}{=} \PY{p}{(}\PY{n}{energy} \PY{o}{+} \PY{n}{m\PYZus{}particle}\PY{p}{)}\PY{o}{/}\PY{n}{m\PYZus{}particle}
    \PY{n}{beta\PYZus{}2} \PY{o}{=} \PY{l+m+mi}{1} \PY{o}{\PYZhy{}} \PY{n}{m\PYZus{}particle}\PY{o}{*}\PY{o}{*}\PY{l+m+mi}{2}\PY{o}{/}\PY{p}{(}\PY{n}{m\PYZus{}particle} \PY{o}{+} \PY{n}{energy}\PY{p}{)}\PY{o}{*}\PY{o}{*}\PY{l+m+mi}{2}
    \PY{k}{return} \PY{n}{xi\PYZus{}func}\PY{p}{(}\PY{n}{energy}\PY{p}{)} \PY{o}{/} \PY{n}{eps\PYZus{}max\PYZus{}func}\PY{p}{(}\PY{n}{energy}\PY{p}{)} \PY{o}{\PYZhy{}} \PY{n}{kappa\PYZus{}value}
\end{Verbatim}
\end{tcolorbox}

    \begin{tcolorbox}[breakable, size=fbox, boxrule=1pt, pad at break*=1mm,colback=cellbackground, colframe=cellborder]
\prompt{In}{incolor}{5}{\boxspacing}
\begin{Verbatim}[commandchars=\\\{\}]
\PY{n}{xi} \PY{o}{=} \PY{p}{(}\PY{l+m+mi}{1}\PY{o}{/}\PY{l+m+mi}{2} \PY{o}{*} \PY{n}{K}\PY{p}{)} \PY{o}{*} \PY{p}{(}\PY{n}{Z} \PY{o}{/} \PY{n}{A}\PY{p}{)} \PY{o}{*} \PY{p}{(}\PY{n}{s} \PY{o}{/} \PY{n}{beta\PYZus{}2}\PY{p}{)} \PY{c+c1}{\PYZsh{} MeV}

\PY{n}{epsilon\PYZus{}max} \PY{o}{=} \PY{n}{eps\PYZus{}max\PYZus{}func}\PY{p}{(}\PY{n}{energy}\PY{p}{)} \PY{c+c1}{\PYZsh{} MeV}

\PY{n}{kappa} \PY{o}{=} \PY{n}{xi} \PY{o}{/} \PY{n}{epsilon\PYZus{}max}

\PY{n+nb}{print}\PY{p}{(}\PY{l+s+s2}{\PYZdq{}}\PY{l+s+s2}{k =}\PY{l+s+s2}{\PYZdq{}}\PY{p}{,}\PY{n}{kappa}\PY{p}{)}
\end{Verbatim}
\end{tcolorbox}

    \begin{Verbatim}[commandchars=\\\{\}]
k = 5.98637846070628e-08
    \end{Verbatim}

    \hypertarget{comparison-with-calculations-from-httpsnap.nationalacademies.orgread20066chapter10189}{%
\subsection{Comparison with calculations from \cite{NAP20066} }\label{comparison-with-calculations-from-httpsnap.nationalacademies.orgread20066chapter10189}}

    \begin{tcolorbox}[breakable, size=fbox, boxrule=1pt, pad at break*=1mm,colback=cellbackground, colframe=cellborder]
\prompt{In}{incolor}{6}{\boxspacing}
\begin{Verbatim}[commandchars=\\\{\}]
\PY{n}{Image}\PY{p}{(}\PY{l+s+s2}{\PYZdq{}}\PY{l+s+s2}{epsilon and xi formulas.png}\PY{l+s+s2}{\PYZdq{}}\PY{p}{,} \PY{n}{width}\PY{o}{=}\PY{l+m+mi}{1000}\PY{p}{)}
\end{Verbatim}
\end{tcolorbox}
 
            
\prompt{Out}{outcolor}{6}{}
    
    \begin{center}
    \adjustimage{max size={0.9\linewidth}{0.9\paperheight}}{Chapters/Vavilov_Landau/output_10_0.png}
    \end{center}
    { \hspace*{\fill} \\}
    

    Then follows a table with the values calculated for different energies:

    \begin{tcolorbox}[breakable, size=fbox, boxrule=1pt, pad at break*=1mm,colback=cellbackground, colframe=cellborder]
\prompt{In}{incolor}{7}{\boxspacing}
\begin{Verbatim}[commandchars=\\\{\}]
\PY{n}{Image}\PY{p}{(}\PY{l+s+s2}{\PYZdq{}}\PY{l+s+s2}{Table of k and xi values.png}\PY{l+s+s2}{\PYZdq{}}\PY{p}{,} \PY{n}{width}\PY{o}{=}\PY{l+m+mi}{800}\PY{p}{)}
\end{Verbatim}
\end{tcolorbox}
 
            
\prompt{Out}{outcolor}{7}{}
    
    \begin{center}
    \adjustimage{max size={0.9\linewidth}{0.9\paperheight}}{Chapters/Vavilov_Landau/output_12_0.png}
    \end{center}
    { \hspace*{\fill} \\}
    

    Comparison of these values with my previous calculations

    \begin{tcolorbox}[breakable, size=fbox, boxrule=1pt, pad at break*=1mm,colback=cellbackground, colframe=cellborder]
\prompt{In}{incolor}{8}{\boxspacing}
\begin{Verbatim}[commandchars=\\\{\}]
\PY{c+c1}{\PYZsh{} Values from 100 MeV to 10 GeV}
\PY{n}{energy\PYZus{}array} \PY{o}{=} \PY{n}{np}\PY{o}{.}\PY{n}{array}\PY{p}{(}\PY{p}{[}\PY{l+m+mi}{100}\PY{p}{,}\PY{l+m+mi}{150}\PY{p}{,}\PY{l+m+mi}{200}\PY{p}{,}\PY{l+m+mi}{300}\PY{p}{,}\PY{l+m+mi}{400}\PY{p}{,}\PY{l+m+mi}{500}\PY{p}{,}\PY{l+m+mi}{600}\PY{p}{,}\PY{l+m+mi}{800}\PY{p}{,}\PY{l+m+mi}{1000}\PY{p}{,}\PY{l+m+mi}{1500}\PY{p}{,}\PY{l+m+mi}{2000}\PY{p}{,}\PY{l+m+mi}{3000}\PY{p}{,}\PY{l+m+mi}{4000}\PY{p}{,}\PY{l+m+mi}{5000}\PY{p}{,}\PY{l+m+mi}{6000}\PY{p}{,}\PY{l+m+mi}{8000}\PY{p}{,}\PY{l+m+mi}{10000}\PY{p}{]}\PY{p}{)}
\end{Verbatim}
\end{tcolorbox}

    \begin{tcolorbox}[breakable, size=fbox, boxrule=1pt, pad at break*=1mm,colback=cellbackground, colframe=cellborder]
\prompt{In}{incolor}{9}{\boxspacing}
\begin{Verbatim}[commandchars=\\\{\}]
\PY{n}{table\PYZus{}data} \PY{o}{=} \PY{n}{np}\PY{o}{.}\PY{n}{vstack}\PY{p}{(}\PY{p}{(}\PY{l+m+mi}{1} \PY{o}{\PYZhy{}} \PY{n}{m\PYZus{}particle}\PY{o}{*}\PY{o}{*}\PY{l+m+mi}{2} \PY{o}{/} \PY{p}{(}\PY{n}{m\PYZus{}particle}\PY{o}{+}\PY{n}{energy\PYZus{}array}\PY{p}{)}\PY{o}{*}\PY{o}{*}\PY{l+m+mi}{2}\PY{p}{,}
              \PY{n}{eps\PYZus{}max\PYZus{}func}\PY{p}{(}\PY{n}{energy\PYZus{}array}\PY{p}{)}\PY{p}{,}
              \PY{n}{A}\PY{o}{/}\PY{n}{Z} \PY{o}{*} \PY{n}{xi\PYZus{}func}\PY{p}{(}\PY{n}{energy\PYZus{}array}\PY{p}{)} \PY{o}{/} \PY{n}{s}\PY{p}{,}
              \PY{n}{A}\PY{o}{/}\PY{n}{Z} \PY{o}{*} \PY{n}{kappa\PYZus{}func}\PY{p}{(}\PY{n}{energy\PYZus{}array}\PY{p}{,}\PY{n}{kappa\PYZus{}value}\PY{o}{=}\PY{l+m+mi}{0}\PY{p}{)} \PY{o}{/} \PY{n}{s}\PY{p}{)}\PY{p}{)}

\PY{n}{df} \PY{o}{=} \PY{n}{pd}\PY{o}{.}\PY{n}{DataFrame}\PY{p}{(}\PY{n}{data}\PY{o}{=}\PY{n}{table\PYZus{}data}\PY{o}{.}\PY{n}{transpose}\PY{p}{(}\PY{p}{)}\PY{p}{,} \PY{n}{columns}\PY{o}{=}\PY{p}{(}\PY{l+s+s2}{\PYZdq{}}\PY{l+s+s2}{\PYZdl{}}\PY{l+s+se}{\PYZbs{}\PYZbs{}}\PY{l+s+s2}{beta \PYZca{}2\PYZdl{}}\PY{l+s+s2}{\PYZdq{}}\PY{p}{,}\PY{l+s+s2}{\PYZdq{}}\PY{l+s+s2}{\PYZdl{}}\PY{l+s+s2}{\PYZbs{}}\PY{l+s+s2}{epsilon\PYZus{}}\PY{l+s+si}{\PYZob{}max\PYZcb{}}\PY{l+s+s2}{\PYZdl{}}\PY{l+s+s2}{\PYZdq{}}\PY{p}{,}\PY{l+s+s2}{\PYZdq{}}\PY{l+s+s2}{\PYZdl{}A/Z*}\PY{l+s+se}{\PYZbs{}\PYZbs{}}\PY{l+s+s2}{xi/s\PYZdl{}}\PY{l+s+s2}{\PYZdq{}}\PY{p}{,}\PY{l+s+s2}{\PYZdq{}}\PY{l+s+s2}{\PYZdl{}A/Z*}\PY{l+s+s2}{\PYZbs{}}\PY{l+s+s2}{kappa/s\PYZdl{}}\PY{l+s+s2}{\PYZdq{}}\PY{p}{)}\PY{p}{)}
\PY{n}{df}\PY{o}{.}\PY{n}{insert}\PY{p}{(}\PY{l+m+mi}{0}\PY{p}{,} \PY{l+s+s2}{\PYZdq{}}\PY{l+s+s2}{Energy}\PY{l+s+s2}{\PYZdq{}}\PY{p}{,} \PY{n}{pd}\PY{o}{.}\PY{n}{DataFrame}\PY{p}{(}\PY{n}{energy\PYZus{}array}\PY{p}{,}\PY{n}{dtype}\PY{o}{=}\PY{n}{np}\PY{o}{.}\PY{n}{int\PYZus{}}\PY{p}{)}\PY{p}{)}

\PY{n}{display}\PY{p}{(}\PY{n}{df}\PY{p}{)}
\end{Verbatim}
\end{tcolorbox}

    
    \begin{Verbatim}[commandchars=\\\{\}]
    Energy  \$\textbackslash{}beta \^{}2\$  \$\textbackslash{}epsilon\_\{max\}\$  \$A/Z*\textbackslash{}xi/s\$  \$A/Z*\textbackslash{}kappa/s\$
0      100  1.8336E-01        2.2919E-01   8.3735E-01      3.6534E+00
1      150  2.5668E-01        3.5247E-01   5.9816E-01      1.6971E+00
2      200  3.2055E-01        4.8153E-01   4.7898E-01      9.9471E-01
3      300  4.2587E-01        7.5699E-01   3.6053E-01      4.7627E-01
4      400  5.0847E-01        1.0556E+00   3.0196E-01      2.8607E-01
5      500  5.7444E-01        1.3773E+00   2.6728E-01      1.9407E-01
6      600  6.2798E-01        1.7221E+00   2.4450E-01      1.4198E-01
7      800  7.0866E-01        2.4809E+00   2.1666E-01      8.7329E-02
8     1000  7.6569E-01        3.3321E+00   2.0052E-01      6.0178E-02
9     1500  8.5193E-01        5.8636E+00   1.8022E-01      3.0736E-02
10    2000  8.9804E-01        8.9708E+00   1.7097E-01      1.9059E-02
11    3000  9.4324E-01        1.6908E+01   1.6278E-01      9.6272E-03
12    4000  9.6390E-01        2.7135E+01   1.5929E-01      5.8701E-03
13    5000  9.7504E-01        3.9646E+01   1.5747E-01      3.9719E-03
14    6000  9.8171E-01        5.4431E+01   1.5640E-01      2.8733E-03
15    8000  9.8898E-01        9.0793E+01   1.5525E-01      1.7099E-03
16   10000  9.9264E-01        1.3616E+02   1.5468E-01      1.1360E-03
    \end{Verbatim}

    
    \hypertarget{at-what-energy-gamma-the-value-kappa-0.01}{%
\subsection{\texorpdfstring{At what energy (\(\gamma\)) the value
\(\kappa = 0.01\)}{At what energy (\textbackslash gamma) the value \textbackslash kappa = 0.01}}\label{at-what-energy-gamma-the-value-kappa-0.01}}

    \begin{tcolorbox}[breakable, size=fbox, boxrule=1pt, pad at break*=1mm,colback=cellbackground, colframe=cellborder]
\prompt{In}{incolor}{10}{\boxspacing}
\begin{Verbatim}[commandchars=\\\{\}]
\PY{n}{kappa\PYZus{}value} \PY{o}{=} \PY{l+m+mf}{0.01}

\PY{n}{target\PYZus{}energy} \PY{o}{=} \PY{n}{fsolve}\PY{p}{(}\PY{n}{kappa\PYZus{}func}\PY{p}{,} \PY{n}{x0}\PY{o}{=}\PY{l+m+mi}{500}\PY{p}{,} \PY{n}{args}\PY{o}{=}\PY{p}{(}\PY{n}{kappa\PYZus{}value}\PY{p}{)}\PY{p}{)}
\PY{n+nb}{print}\PY{p}{(}\PY{n}{target\PYZus{}energy}\PY{p}{[}\PY{l+m+mi}{0}\PY{p}{]}\PY{p}{,} \PY{l+s+s2}{\PYZdq{}}\PY{l+s+s2}{MeV}\PY{l+s+s2}{\PYZdq{}}\PY{p}{)}
\end{Verbatim}
\end{tcolorbox}

    \begin{Verbatim}[commandchars=\\\{\}]
148.80746742515953 MeV
    \end{Verbatim}


    % Add a bibliography block to the postdoc
    


%%% I need to replace the \si with \qty or \num or \unit

The theoretical distribution of the energy loss of heavy particles hitting a thin target has been solved rigorously by Vavilov in \cite{vavilov_1957}. The namesake distribution is a generalization of the Landau distribution, but its evaluation is more difficult, as it is expressed as an integral over some complicated functions \cite[Eq.(4)]{vavilov_1957}. Fortunately, it can be approximated by more straightforward distributions depending on the value of a parameter $\kappa$ (Eq. \eqref{eq:kappa_definition}): for $\kappa\rightarrow0$ the Vavilov distribution can be approximated by a Langau, for $\kappa>>1$ by a Gaussian. 
In this section we have replicated the calculations done in \cite{NAP20066} applied to our specific case to ensure that the Landau approximation is valid.

% \subsection{The \(\kappa\) factor}
\subsection{Vavilov approximation parameter} % remove $\kappa$ because of hyperref for now

The parameter \(\kappa\) is defined as:

\begin{equation}\label{eq:kappa_definition}
    \kappa = \frac{\xi}{\epsilon_{max}}
\end{equation}

where $\xi$ and $\epsilon_{max}$ \cite[Eq.(1)]{NAP20066} are:

\begin{equation*}
    \xi = \frac{1}{2} K \frac{Z}{A} \frac{s} {\beta^2}; \qquad s = \Delta x \cdot \rho
\end{equation*}

\begin{equation*}
\epsilon_{max} = \frac{2m_e c^2 \beta^2}{1-\beta^2}\left[ 1 + \frac{2m_e}{M}\frac{1}{\sqrt{1-\beta^2}} + \left(\frac{m_e}{M}\right)^2 \right]^{-1}
\end{equation*}


% \subsection[Vavilov distribution factor]{Calculating the $\kappa$ factor}\label{vavilov_distribution_factor}
The parameters just mentioned are the same that appear in the Bethe-Bloch mean energy loss:
\begin{equation*}
    \left\langle-\frac{dE}{dx} \right\rangle \frac{1}{\rho} =  K z_0^2 \frac{1}{\beta^2} \frac{Z}{A M_u} \cdot\left[\frac{1}{2}\ln \left(\frac{2m_e c^2 \beta^2 W_{max}}{I^2 \cdot (1-\beta^2)}\right) - \beta^2\right]
\end{equation*}

And they have the following values:

\begin{equation*}
    \frac{K}{M_u} = 4 \pi N_A r_e^2 m_e c^2 /M_u = 0.307075 \si{MeV.g^{-1}.cm^2}
\end{equation*}


\begin{equation*}
r_e = \frac{e^2}{4\pi\varepsilon_0 m_e c^2}; \qquad
M_u = 1 \frac{\si{g}}{\si{mol}}; \qquad z_0=1; \qquad
\left( W_{max} \equiv \epsilon_{max} \right)
\end{equation*}

For the case of \textbf{silicon}:
\begin{equation*}
    Z = 14; \qquad
    A = 28.085; \qquad
    \rho = 2.329085 \si{g.cm^{-3}}; \qquad
    I = 173 \si{eV} \footnote[1]{Mean excitation Energy}
\end{equation*}

Finally, for a Pion ($M=139.570\si{MeV/c^2}$) with energy $E=120\si{GeV}$ ($\gamma=860.78$ and $\beta^2=0.99999865$) and the thin layer ($50\si{\micro\meter}$) of an LGAD sensor, we obtained a value:

\begin{equation}\label{eq:kappa_value}
    \kappa = \num{8.5959e-09}
\end{equation}

Well within the Landau approximation.

Reversing the calculation and fixing $\kappa=0.01$, which is approximately the lower limit of validity, the energy of the particle should be below: $491.9\si{MeV}$. 




\section{Additonal plots}\label{sec:additional_plots}

\begin{figure}[!ht]
    \centering
    \subfloat[Using pulseHeight cut]{
        \includegraphics[width=.48\textwidth]{Images/appendix/2D_Tracks_401_S1 highlight geometry cut (using pulseHeight).png}
        \label{fig:geometry_cut_using_pulseHeight}}
    \hfill
    \centering
    \subfloat[Using time cut]{
        \includegraphics[width=.48\textwidth]{Images/appendix/2D_Tracks_401_S1 highlight geometry cut (using time).png}
        \label{fig:geometry_cut_using_time}}
    \caption{Example of the two \textit{geometry cuts} (red rectangles) obtained by applying two different cuts}
    \label{fig:geometry_cut_comparison}
\end{figure}

\begin{figure}[!ht]
    \centering
    \includegraphics[width=0.7\linewidth]{Images/appendix/Charge_distribution_different_cuts_batch_401_S1_DUTs_3.png}
    \caption{The different quality cuts applied and their individual effect on the charge distribution}
    \label{fig:charge_plot_all_cuts}
\end{figure}