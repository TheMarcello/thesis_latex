\begin{abstract}
    %%% 150-250 words

    %%% HL-LHC, HGTD in the ATLAS upgrades
    With the future High-Luminosity phase of the LHC, a new detector system will be installed in the ATLAS experiment: the High Granularity Timing Detector. It will provide time information of charged particles, and it will help mitigate the effects of the increased pile up that the higher luminosity will bring. Low Gain Avalanche Detectors will compose the modules of the HGTD.
    %%% LGADs must meet requirements, charge, efficiency, time resolution
    To meet the performance requirements, LGADs will be expected to have a collected charge of at least \qty{4}{\femto\coulomb}, an efficiency of \(95\%\), and a time resolution of \qty{35}{\pico\second} (\qty{70}{\pico\second}) at the beginning (end) of their life. 

    %%% test beam campaign of May 2023 at CERN etc.
    %%% three different vendors and various radiation doses
    This thesis presents an anaysis of the May 2023 test beam campaign, conducted at CERN, using a \qty{120}{\giga\electronvolt} pions beam provided by SPS. Devices from three different vendors were tested: IHEP-IME, USTC and CNM. The sensors were irradiated up to a fluence of \qty{2.5e15}{\neutroneq}, corresponding to the total expected dose during operation.
    %%% results: performance of the sensors
    All sensors irradiated up to \qty{1.5e15}{\neutroneq} satisfied the HGTD's requirements for bias voltages lower than \qty{400}{\volt}. At the highest fluence, the LGADs did not meet the charge and efficiency requirements, but still achieved a time resolution below \qty{70}{\pico\second}. Additionally, the sensors were tested at angles of \qtyrange{0}{14}{\degree} to the beam line, which showed a small consistent improvement in charge collection and time resolution. 
 
\end{abstract}