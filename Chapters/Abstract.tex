\begin{abstract}

    %%% 150-250 words

    %%% HL-LHC, HGTD in the ATLAS upgrades
    In the upcoming High-Luminosity phase of the LHC, the ATLAS experiment will be upgraded with a new detector system: the High Granularity Timing Detector. The HGTD will be the first large-scale detector to provide timing information to ATLAS, deploying a new technology: Low Gain Avalanche Detectors (LGAD), thin silicon diodes with a mild internal charge multiplication.
    % It will provide time information of charged particles, and it will help mitigate the effects of the increased pileup that the higher luminosity will bring. The HGTD will be composed of modules whose active elements will be Low Gain Avalanche Detectors: thin silicon detectors with an internal charge amplification.
    %%% LGADs must meet requirements, charge, efficiency, time resolution
    According to the HGTD's designed performance, LGADs are required to have a collected charge of at least \qty{4}{\femto\coulomb}, an efficiency of \(95\%\), and a time resolution of \qty{35}{\pico\second} (\qty{70}{\pico\second}) at the beginning (end) of their life. 

    %%% test beam campaign of May 2023 at CERN etc.
    %%% three different vendors and various radiation doses
    This thesis presents an analysis of the May 2023 test beam campaign, conducted at CERN, using a \qty{120}{\giga\electronvolt} pions beam provided by SPS. Devices from three different vendors were tested: IHEP-IME, USTC and CNM. The devices were tested at \qty{-30}{\degreeCelsius}, the temperature of the HGTD in operation. The sensors were irradiated to different fluences, up to \qty{2.5e15}{\neutroneq}, corresponding to the maximum expected dose for a single sensor during its time in operation.

    %%% results: performance of the sensors
    All sensors irradiated up to \qty{1.5e15}{\neutroneq} satisfied the HGTD's requirements at bias voltages lower than \qty{400}{\volt}. At the highest fluence, the LGADs did not meet the charge and efficiency requirements, but still achieved a time resolution below \qty{70}{\pico\second}. Additionally, the sensors were tested at angles of \qtyrange{0}{14}{\degree} to the beam line, which showed a small but consistent improvement in charge collection and time resolution.
 

\end{abstract}
