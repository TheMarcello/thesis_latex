%%% what LGADs are
%%% working principles
%%% requirements for the detector
%%% other studies 
%%% expected results
\chapter{LGADs}\label{chap:LGADs}

\section{The High Granularity Time Detector}\label{sec:HGTD}

As mentioned before in \nameref{subsec:high_luminosity_upgrade}, the HL-LHC will see an increase in the number of collisions per bunch\footref{footnote:particle_beam_bunches}, with heightened pile-up effects (when a collision happens very close to one or more other events, thus complicating track reconstruction.) 
 
A new way to mitigate this effect will be "\textit{to use high-precision timing information to distinguish between collisions occurring very close in space but well-separated in time}" \cite{CERN-LHCC-2020-007}. This is set to be accomplished by the High Granularity Time Detector (HGTD).
 
% eta=2.4 -> theta=10.367°      eta=4 -> theta=2.0986°
The detector will cover pseudrapidities between $2.4 < \eta < 4.0$ \footnote{the pseudorapity is $\eta=-\ln \tan(\theta/2)$ where $\theta$ is the polar angle from the $z$ axis, i.e. the beam direction.} (so an incident angle roughly between 2° and 10°), complementing the ITk (Inner Tracker)\marginpar{\flushleft I need to talk about ITk and what it will be replacing \nameref{subsec:ATLAS_upgrades}} which has worse resolution in this forward region. %%% I could put plot with the simulation of z resolution vs \eta of ITk
% picture of HGTD
\begin{figure}[!ht]
    \centering
    \subfloat[The detector will be placed outside the volume of the future Inner Tracker, between the Barrel ande Forward Calorimeters]{
        \includegraphics[width=.6\linewidth]{Images/intro/HGTD_position_and_layout.png}
        \label{fig:HGTD_location}}
    \hfill
    \centering
    \subfloat[The overlap between modules on the front and the back was optimized to give an approximately uniform performance (as a function of radius) \cite{CERN-LHCC-2020-007}.]{
        \includegraphics[width=.35\linewidth]{Images/intro/HGTD_disks_alignment.png}
        \label{fig:HGTD_schema}}
    \caption{Location (\ref{fig:HGTD_location}) and of the HGTD inside ATLAS and orientation (\ref{fig:HGTD_schema})of the two faces on each disk. From \cite{cernTechnicalDesign}.}
\end{figure}

\subsection{Detector layout}\label{subsec:HGTD_layout}

The detector consists of two thin disks with that will be placed outside the ITk , as shown in Figure \ref{fig:HGTD_location}. The detector is divided into three concentric active regions, as can be seen Figures \ref{fig:HGTD_location} and \ref{fig:pulseHeight_cut}, ranging: \qty{120}{\milli\meter} - \qty{230}{\milli\meter}, \qty{230}{\milli\meter} - \qty{470}{\milli\meter} and  \qty{470}{\milli\meter} - \qty{640}{\milli\meter} \cite{CERN-LHCC-2020-007}. Beyond the last region aret the peripheral electronics. The two disks are covered by 8032 modules in total, each module containing two silicon sensors and two ASICs (Application-Specific Integrated Circuit). Each sensor features 15x15 pads of Low Gain Avalanche Detectors (LGADs, discussed in the next Chapter \ref{chap:LGADs}).

The radiation that the components will withstand depends strongly on the radius. Due to electronics and sensors specifications, a minimum charge of \qty{4}{\femto\coulomb} is required. This is achievable up to radiation damage of \qty{2.5e15}{\neutroneq\centi\meter^{-2}}\footnote{The radiation damage equivalent to neutrons at \qty{1}{\mega\electronvolt} }; as a result, the sensor and electronics within the smallest ring will be replaced after each \qty{1000}{\femto\barn^{-1}} and the ones within the middle ring will be replaced at half of the data-taking (\qty{20000}{\femto\barn^{-1}}). 

% The luminous region in the nominal scenario for Run 3 will see a Gaussian spread of approximately 50mm along the $z$ axis and a width of 175$\si{ps}$ in time %(Figure \ref{fig:events_pileup}).

HGTD will provide time information of charged particles with a time resolution of 30ps (up to 50ps at the end of life) which will greatly improve the reconstruction of the primary vertex of collision. 

% \begin{figure}[!hb]
%     \centering
%     \includegraphics[width=.8\textwidth]{Images/intro/events_pileup_HL_LHC.jpg}
%     \caption{Visualization of the pile-up of events inside ATLAS, using simulation data (PUT SOURCE AND MORE EXPLANATIONS (what is HS?))}
%     \label{fig:events_pileup}
% \end{figure}
\marginpar{\flushleft Add other image/photo of LGAD}


\section{Working principles of silicon sensors}

The core of a silicon sensor consists of a junction between two differently doped layers, which means that small concentrations of impurities with either higher atomic number ($n$-type) or lower ($p$-type) are introduced inside the crystals.
This forms a $pn$-junction and when a voltage is applied with positive potential on the $n$-side and negative on the $p$-side (reverse bias) the volume between the two layers is depleted of mobile charges and becomes an insulator with an internal electric field.

% %%% FIGURE WITH SIDE CAPTION
% \sidecaptionvpos{figure}{c}
% \begin{SCfigure}
%     \includegraphics[width=.4\linewidth]{Images/LGADs/p-n junction with voltage.png}
%     \caption{Top: adjacent regions of $p$-doping (left) and $n$-doping forming a $pn$-junction. Middle: the circled mobile charges (holes for $p$-type and electrons for $n$-type) balanced by the charge of atomic cores. Bottom: When an external (reverse) voltage is applied to the central region an electric field builds up in the junction.}
% \end{SCfigure}

%%% WRAPPED FIGURE
% \begin{wrapfigure}{l}{.45\linewidth}
%     \includegraphics[width=1\linewidth]{Images/LGADs/p-n junction with voltage.png}
%     \caption{Top: adjacent regions of $p$-doping (left) and $n$-doping forming a $pn$-junction. Middle: the circled mobile charges (holes for $p$-type and electrons for $n$-type) balanced by the charge of atomic cores. Bottom: When an external (reverse) voltage is applied to the central region an electric field builds up in the junction.}
%     \label{fig:p-n_junction_reverse_bias_voltage}
% \end{wrapfigure}

%%% FIGURE WITH MINIPAGE
\begin{figure}[!h]
    \begin{minipage}[c]{.25\linewidth}
        \includegraphics[width=1\linewidth]{Images/LGADs/p-n junction with voltage.png}
    \end{minipage}
    \hfill
    \begin{minipage}[c]{.6\linewidth}
        \caption{\\Top: adjacent regions of $p$-doping (left) and $n$-doping (right) forming a $pn$-junction.\\
        Middle: the circled mobile charges (holes for $p$-type and electrons for $n$-type) balanced by the charge of atomic cores.\\
        Bottom: When an external (reverse) voltage is applied to the junction an electric field builds up in the central region \cite{10.1093/acprof:oso/9780198527848.003.0001}.}
    \end{minipage}
    \label{fig:p-n_junction_reverse_bias_voltage}
\end{figure} 

When a charged particle traverses this depletion layer it frees up electron-hole pairs, which move to the electrodes and can be measured. 

\section{Low Gain Avalanche Detectors}

A particular type of silicon sensors are Low Gain Avalanche Detectors (LGAD), an example is shown is shown in Figure \ref{fig:LGADs_schema}. The major innovation is an additional $p$-type layer below the $n+$ electrode, this creates a high electric field region which leads to an avalanche effect\footnote[2]{When electrons acquire enough energy they can create new electron-hole pairs ('impact ionization'), which can themselves create new pairs and initialize a multiplication chain that leads to an enhanced signal} of the electrons. This effect produces a gain of around $~10$ \marginpar{\flushleft source} 

\begin{figure}[!ht]
    \centering
    \includegraphics[width=.9\linewidth]{Images/LGADs/LGADs_schema_of_work.png}
    \caption{Cut view of an LGAD, not to scale.}
    \label{fig:LGADs_schema}
\end{figure}





%%% NOT UPDATED ANYMORE, LOOK AT THE EXCEL FILE, I MIGHT DELETE THIS LATER
%%% I have to reorganize these and associate them with the more accurate descriptions
% device name:        vendor:        sensor ID:            fluence:    irradiation type:    type:        board name:    channels:
% CNM-W4              CNM          CNM-R15973-W4-D168      unirradiated      -             single pad     JSI-B12      1
% CNM-W5              CNM          CNM-R15973-W5-D138      unirradiated      -             single pad     JSI-B14      1
% CNM-W3-2.5E15       CNM          CNM-R15973-W3-D29       $\num{2.5e15}     neutron       single pad     JSI B5       1
% CNM-W5-1.5E15       CNM          CNM-R15973-W5-D29       $\num{1.5e15}     neutron       single pad     JSI PP1      1
% USTC2.1             USTC         USTC2.1-W17-P6-A          0                -             2x2           CERN-3       1,2
% USTC2.1 IRRADIATED (MISSING)
% IMEv3-W12-C2         IHEP        IMEv3-W12-C2-2-2          0               -              2x2          CERN-1       channels 1,2
% IMEv3-W12-C3         IHEP        IMEv3-W12-C3-1-4 (and 5)  0               -              1x3          CERN-1       channles 3,4  (small GR), bonded
% CERN2-CH0-IMEv3-W12  IHEP        IMEv3-W12-B2-2-9-1       1.5e15           neutron        2x2 sensor    CERN-2       channels 1,2,3
% CERN2-CH1-IMEv3-W12  IHEP 
% CERN2-CH2-IMEv3-W12  IHEP 
% CERN2-CH4-IMEv3-W16  IHEP        IMEv3-W16-Q4-D4-1-4      1.5e15           neutron        1x3          CERN-2       channel:  2(?)
% JSI-B6-IMEv2-W7-1E14    IHEP       W7-II-C2-1-7 IMEv2-W7Q2    1e14          proton        single       JSI-B6
% JSI-PP4-IMEv2-W7-6.5E14 IHEP       W7-II-C2-1-7 IMEv2-W7Q2    6.5e14        proton        single       JSI-PP4
% JSI-B7-IMEv3-W16-8E14   IHEP       IHEP-IMEv3-W16_Q4_D3_1-4   8e14          (unsure)        single       JSI-B7
% JSI-B13-IMEv3-W16-2.5E15 IHEP      IHEP-IMEv3-W16_Q4_E3_1-4   2.5e15       (unsure)      single        JSI-B13
                 