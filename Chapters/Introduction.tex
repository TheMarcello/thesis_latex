
\chapter*{Introduction}\label{chap:intro}
\addcontentsline{toc}{chapter}{\nameref{chap:intro}}

%%%     - LHC high lumi, pileups in ATLAS
%%%     - HGTD and LGADs 
In the coming years, the LHC will undergo a series of upgrades to accommodate the requirements of the High-Luminosity phase~\cite{cernHLLHCProject}, aimed at enabling more precise measurements and improving sensitivity to rare processes beyond the Standard Model.

In particular, the ATLAS experiment will have many of its subdetectors improved and/or replaced. Additionally, to help deal with the problems caused by the increased luminosity, a new detector system will be integrated into the experiment: the High Granularity Timing Detector~\cite{cernTechnicalDesign}. The HGTD will provide time information of charged particles, helping differentiate between collisions that happen very close in space but well separated in time. To achieve this, a special type of silicon sensors will be employed: Low Gain Avalanche Detectors~\cite{PELLEGRINI201412}. These sensors combine very fast response with an internal gain, to enhance the signal and better endure the high radiation environment that they will face.

%%%     - LGADs requirements
%%%     - test beam
To achieve the HGTD's precision targets, the LGADs need to satisfy some requirements. The sensors must collect enough charge for the signal to be measured by the electronics (\(\approx \qty{4}{\femto\coulomb} \)), they must have a time resolution better than \qty{35}{\pico\second} at start of life (and up to \qty{70}{\pico\second} at end of life), and, finally, they must have an efficiency per hit greater than \(95\%\).

There exist several vendors and version of LGADs so, to test and compare their perfomances, the sensors are typically put through a collimated beam of accelerated particles, simulating the environment of the ATLAS detector. In this thesis we focused on the test beam campaign of May 2023, conducted at one of CERN's facilities, where LGADs from three different vendors were characterized: USTC (University of Science and Technology of China), IHEP (Institute of High Energy Physics, China) and CNM (Centro Nacional de Microelectr\'onica, Spain).

In the test beam a telescope of 6 silicon pixel detectors was used for track reconstruction, a cooling box maintained the LGADs at \qty{-30}{\degreeCelsius} and another device had the role of reference for the time resolution measurements. 

%%% methods overview
After some pre-processing, the data was further refined to suppress various sources of noise. For each sensor, the time resolution, the collected charge and the efficiency were calculated. The sensors were grouped based on version, vendor and level of radiation for easier comparison. Additionally, some interesting effects that emerged during the analysis were further investigated.

%%% Thesis structure (optional)
The \hyperref[chap:LHC_ATLAS]{first chapter} of this thesis describes in general the LHC and the ATLAS experiment, the \hyperref[chap:HGTD_LGADs]{next chapter} focuses on the HGTD and the silicon sensors it will use: the LGAD. The \hyperref[chap:testbeam_setup]{following chapter} describes the key elements in the test beam setup used for the data taking. The \hyperref[chap:analysis]{fourth chapter} follows the implementation of the analysis, providing the reasons and the tools used. Finally, the \hyperref[chap:results]{fifth chapter} presents the general results of the analysis. Some additional information, plots and tables are provided in the \hyperref[chap:appendix]{Appendix}.

