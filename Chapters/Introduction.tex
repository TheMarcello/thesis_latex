
\chapter*{Introduction}\label{chap:intro}
\addcontentsline{toc}{chapter}{\nameref{chap:intro}}

%%% Motivation/baground: WHY
%%%     - LHC high lumi, pileups in ATLAS
%%%     - HGTD and LGADs 
In the upcoming years the LHC will undergo a series of upgrades to accommodate the requirements of the High-Luminosity phase~\cite{cernHLLHCProject}, which will increase the number of instantaneous collisions. In particular, the ATLAS experiment will have many of its subdetectors improved and/or replaced. Additionally, to help deal with the increased pileup effects, a whole new detector will be installed: the High Granularity Time Detector~\cite{cernTechnicalDesign}. The HGTD will provide time information of charged particles, helping differentiate collisions that happen very close in space but well separated in time. To achieve this, a special type of silicon sensors will be employed: Low Gain Avalanche Detectors~\cite{PELLEGRINI201412}. These sensors combine very fast response with an internal gain, to enhance the signal and better endure the high radiation environment that they will face.

%%% Scientific and technical context
%%%     - LGADs requirements
%%%     - test beam
The LGADs need to satisfy some parameters to achieve the HGTD's precision targets. The sensors must collect enough charge for the signal to be measured by the electronics (\(\approx \qty{4}{\femto\coulomb} \)), they must have a time resolution better than \qty{35}{\pico\second} at start of life (up to \qty{70}{\pico\second} at end of life), and finally they must have an efficiency per hit greater than \(95\%\). 

There exist several vendors and version of LGADs so, to test and compare the perfomances of the sensors, they are typically put through a collimated beam of accelerated particles and the signals are recorded and analysed. In this thesis we focused on the test beam campaign of May 2023, conducted at one of CERN's facilities, where LGADs from three different vendors were characterized: USTC (University of Science and Technology of China), IHEP (Institute of High Energy Physics, China) and CNM (Centro Nacional de Microelectr\'onica, Spain).  

%%% Specific objective of the thesis
%%%     -TODO: current LGADs developements?


%%% Method overview
%%%     -

%%% Thesis structure (optional)


